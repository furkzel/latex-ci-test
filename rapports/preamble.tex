% ============================================
% PRÉAMBULE - Configuration LaTeX du Projet
% ============================================
% Ce fichier contient tous les packages et configurations LaTeX.

% ============================================
% MISE EN PAGE
% ============================================
% Marges personnalisées avec plus d'espace à gauche (3cm) pour la reliure
\usepackage[
    a4paper,
    top=2cm,
    bottom=2cm,
    left=3cm,
    right=2cm,
    marginparwidth=0pt
]{geometry}

% ============================================
% PACKAGES GRAPHIQUES ET FIGURES
% ============================================
\usepackage{graphicx}
\usepackage{subcaption}
\usepackage{color}
\usepackage{xcolor}
\usepackage{float}
\usepackage{tikz} % Pour créer des graphiques vectoriels, diagrammes et schémas personnalisés
\usepackage{pgfplots} % Pour générer des graphiques de fonctions et des tracés scientifiques
\pgfplotsset{compat=1.18}

% Chemin des figures
\graphicspath{{tex/figures/}} % Définit le répertoire par défaut pour les images

% ============================================
% PACKAGES POUR LE CODE SOURCE
% ============================================
\usepackage{listings}
\usepackage{algorithm} % Pour créer des environnements flottants pour les algorithmes
\usepackage[noend]{algpseudocode} % Pour écrire des algorithmes en pseudocode formel

% Configuration de listings pour Python
% We can use `minted' for syntax highlighting, but here is a basic setup with `listings` not used yet, but kept for potential future use
\lstset{
language=Python,
basicstyle=\ttfamily\small,
keywordstyle=\color{blue}\bfseries,
commentstyle=\color{gray}\itshape,
stringstyle=\color{red},
numbers=left,
numberstyle=\tiny\color{gray},
stepnumber=1,
numbersep=8pt,
backgroundcolor=\color{white},
showspaces=false,
showstringspaces=false,
showtabs=false,
frame=single,
rulecolor=\color{black},
tabsize=4,
captionpos=b,
breaklines=true,
breakatwhitespace=false,
escapeinside={\%*}{*)},
inputencoding=utf8,
extendedchars=true,
literate={é}{{\'e}}1 {è}{{\`e}}1 {ê}{{\^e}}1 {à}{{\`a}}1 {ù}{{\`u}}1 {û}{{\^u}}1 {ô}{{\^o}}1 {ï}{{\"i}}1 {ç}{{\c{c}}}1 % Support des caractères accentués français dans le code
}

% ============================================
% PACKAGES MATHÉMATIQUES
% ============================================
\usepackage{amsmath, amsfonts, amssymb, amsthm}

% ============================================
% ENCODAGE ET POLICE
% ============================================
\usepackage{libertine}
\usepackage[utf8]{inputenc}
\usepackage[T1]{fontenc}
\usepackage{textcomp} % Support pour les ligatures œ, Œ et autres caractères spéciaux
\usepackage[french]{babel} % Support de la langue française (césure, mise en forme)
\usepackage{csquotes} % Gestion intelligente des guillemets selon le contexte linguistique

% ============================================
% HYPERLIENS ET RÉFÉRENCES
% ============================================
\usepackage{hyperref}
% Configuration détaillée des hyperliens et métadonnées PDF
\hypersetup{
    colorlinks=true,
    linkcolor=blue,
    filecolor=magenta,
    urlcolor=cyan,
    citecolor=green,
    pdftitle={Thèse de Master - CVaR},
    pdfauthor={Votre Nom},
    bookmarks=true,
    bookmarksopen=true,
    pdfpagemode=UseOutlines
}
\usepackage[normalem]{ulem}

% ============================================
% INDENTATION ET TITRES
% ============================================
\usepackage{indentfirst}
\usepackage{titlesec}
\titleformat{\chapter}[display]
{\normalfont\huge\bfseries}{}{0pt}{\Huge}
\titlespacing*{\chapter}{0pt}{10pt}{20pt}

\setcounter{secnumdepth}{3}

% Contrôle des veuves et orphelines pour améliorer la typographie
\widowpenalty=10000 % Évite les lignes isolées en haut de page
\clubpenalty=10000 % Évite les lignes isolées en bas de page
\raggedbottom % Permet une hauteur de page flexible pour éviter les espaces excessifs

% ============================================
% PACKAGES POUR LISTES
% ============================================
\usepackage{enumitem} % Contrôle avancé de la mise en forme des listes

% ============================================
% PACKAGES DE TABLEAUX
% ============================================
\usepackage{booktabs} % Tableaux de qualité professionnelle avec lignes élégantes
\usepackage{multirow} % Fusion de cellules sur plusieurs lignes
\usepackage{longtable} % Tableaux sur plusieurs pages
\usepackage{array} % Amélioration des colonnes de tableaux
\usepackage{tabularx} % Tableaux avec colonnes à largeur automatique

% ============================================
% PACKAGES POUR CAPTIONS
% ============================================
\usepackage{caption} % Personnalisation des légendes de figures et tableaux

% ============================================
% BIBLIOGRAPHIE
% ============================================
% Utilisation de biblatex avec biber pour une gestion avancée de la bibliographie
\usepackage[
    backend=biber,
    style=numeric,
    sorting=none,
    maxbibnames=99
]{biblatex}
\addbibresource{tex/references/references.bib}

% ============================================
% COMMANDES PERSONNALISÉES
% ============================================
% Commande pour mettre du texte en rouge (notes)
\newcommand{\nt}[1]{\textcolor{red}{#1}}

% Ligne horizontale
\newcommand{\HRule}{\rule{\linewidth}{0.5mm}}

% ============================================
% ENVIRONNEMENTS MATHÉMATIQUES
% ============================================
\newtheorem{theorem}{Th\'{e}or\`{e}me}
\newtheorem{lemma}[theorem]{Lemme}
\newtheorem{proposition}[theorem]{Proposition}
\newtheorem{corollary}[theorem]{Corollaire}

\newtheorem{definition}{D\'{e}finition}
\newtheorem{conjecture}{Conjecture}
\newtheorem{example}{Exemple}
\newtheorem{remark}{Remarque}

\def\proofname{D\'emonstration}

% Environnements compatibles avec annexeA.tex (noms en français)
\newtheorem{theoreme}[theorem]{Th\'{e}or\`{e}me}
\newtheorem{lemme}[theorem]{Lemme}
\newtheorem{corollaire}[theorem]{Corollaire}
\newtheorem{exemple}[definition]{Exemple}
\newtheorem{remarque}[definition]{Remarque}

% ============================================
% COMMANDES MATHÉMATIQUES PERSONNALISÉES
% ============================================
\newcommand{\N}{\mathbf{N}}     % Entiers naturels
\newcommand{\R}{\mathbf{R}}     % Réels
\newcommand{\C}{\mathbf{C}}     % Complexes
\newcommand{\Z}{\mathbf{Z}}     % Entiers relatifs
\newcommand{\Q}{\mathbf{Q}}     % Rationnels
\newcommand{\E}{\mathbb{E}}     % Espérance mathématique
\newcommand{\Prob}{\mathbb{P}}  % Mesure de probabilité

% Opérateurs mathématiques pour l'analyse de risque
\DeclareMathOperator{\Var}{Var} % Variance
\DeclareMathOperator{\Cov}{Cov} % Covariance
\DeclareMathOperator{\CVaR}{CVaR} % Conditional Value at Risk
\DeclareMathOperator{\VaR}{VaR} % Value at Risk
\DeclareMathOperator*{\argmin}{argmin} % Argument du minimum
\DeclareMathOperator*{\argmax}{argmax} % Argument du maximum

% Autres commandes utiles pour les notations mathématiques
\newcommand{\norm}[1]{\left\lVert#1\right\rVert} % Norme
\newcommand{\abs}[1]{\left\lvert#1\right\rvert} % Valeur absolue
\newcommand{\inner}[2]{\left\langle#1,#2\right\rangle} % Produit scalaire

% ============================================
% RENOMMAGE DES SECTIONS
% ============================================
\renewcommand*\contentsname{Table des Matières}
\renewcommand*\bibname{Références}

% ============================================
% SOME EXTRA CONFIGURATIONS
% ============================================
% ...
